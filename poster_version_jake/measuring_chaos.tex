% This file creates an a0 portrait poster (3 feet wide X 4 feet high)
% Written by Jennifer de Kleine for her poster for the MITACS AGM 2003.
% Based on template file written by Graeme, 2001-03 based on his SOC poster.
% See discussion and documentation at
% <http://www.astro.gla.ac.uk/users/norman/docs/posters/> 
%
% $Id: poster-template-portrait.tex,v 1.2 2002/12/03 11:25:55 norman Exp $


% We switch to portrait mode. This works as advertised.

\documentclass[a0,landscape]{a0poster}  % 4 feet wide X 3 feet high

% You might find the 'draft' option to a0 poster useful if you have
% lots of graphics, because they can take some time to process and
% display. (\documentclass[draft]{a0poster})

% Switch off page numbers on a poster, obviously, and section numbers too.
\pagestyle{empty}
\setcounter{secnumdepth}{0}

% The textpos package is necessary to position textblocks at arbitary 
% places on the page.
\usepackage[absolute]{textpos}
%\usepackage{subfigure}

% Graphics to include graphics. Times is nice on posters, but you
% might want to switch it off and go for CMR fonts.
\usepackage[final]{graphics}
\usepackage{graphicx,wrapfig,amsfonts, eso-pic,url}
\usepackage{amsmath,amsfonts,amssymb,amsthm,mathtools,enumitem,blkarray}
\usepackage[justification=centering]{caption}
\usepackage{algorithmicx,algpseudocode}
\usepackage{stmaryrd}
\algrenewcommand\algorithmicforall{\textbf{for each}}
\algrenewcommand{\algorithmiccomment}[1]{// #1}

%BEGIN GARY TIKZ PRE

%\usepackage[svgnames]{xcolor}
\usepackage{tikz}
%\usetikzlibrary{decorations.markings}
%\pagestyle{empty}

\pgfdeclarelayer{edgelayer}
\pgfdeclarelayer{nodelayer}
\pgfsetlayers{edgelayer,nodelayer,main}

\tikzstyle{none}=[inner sep=0pt]
\definecolor{hexcolor0xf81e1c}{rgb}{0.973,0.118,0.110}
\definecolor{hexcolor0x3c00ff}{rgb}{0.235,0.000,1.000}

\tikzstyle{vertex}=[circle, fill=white,draw=black, scale=0.55]
\tikzstyle{whitevertex}=[circle,fill=white,draw=black, scale = 0.5]
\tikzstyle{redvertex}=[circle,fill=hexcolor0xf81e1c,draw=black, scale = 0.5]
\tikzstyle{bluevertex}=[circle,fill=hexcolor0x3c00ff,draw=black, scale = 0.5]
\tikzstyle{greenvertex}=[circle,fill=green,draw=black, scale=0.5]
\tikzstyle{purplevertex}=[circle,fill=magenta,draw=black, scale=0.5]
\tikzstyle{yellowvertex}=[circle,fill=yellow,draw=black, scale=0.5]
\tikzstyle{grayvertex}=[circle,fill=gray,draw=black, scale=0.5]
\tikzstyle{blackvertex}=[circle,fill=black,draw=black, scale=0.5]
\tikzstyle{cyanvertex}=[circle,fill=cyan,draw=black, scale=0.5]

\tikzstyle{textbox}=[rectangle,fill=none,draw=none]
\tikzstyle{box}=[rectangle,fill=none,draw=black]

\tikzstyle{arc}=[black, ->]
\tikzstyle{grayarc}=[gray, ->]
\tikzstyle{bluearc}=[blue, ->]
\tikzstyle{grayedge}=[draw=gray]
\tikzstyle{blueedge}=[draw=blue, thick]
\tikzstyle{rededge}=[draw=red, thick]
\tikzstyle{purpleedge}=[draw=magenta, very thick]
\tikzstyle{greenedge}=[draw=green, thick]
\tikzstyle{blackedge}=[draw=black, thick]
\tikzstyle{dashededge}=[draw=black, dashed]
\tikzstyle{edge}=[draw=black]

\tikzstyle{10circle}=[circle, scale=10.0,draw=black]
\tikzstyle{10oval}=[ellipse, scale=10.0,draw=black]

%END GARY'S TIKZ PRE


\newcommand{\F}{\mathbb{F}}
\newcommand{\Fm}{\mathbb{F}_{2^m}}
\newcommand{\E}{\mathcal E}
\newcommand{\Z}{\mathbb{Z}}
\newcommand{\Q}{\mathbb{Q}}
\newcommand{\M}{\mathcal M}
\newcommand{\Cone}{\mathcal C}
\newcommand{\N}{\mathbb{N}}
\newcommand{\bm}[1]{\mathbf{#1}}

\def\tr{{\rm Tr}}
\def\Tr{{\rm Tr}}

\newcommand{\R}{\mathbb{R}}
\newcommand{\C}{\mathbb{C}}
\newcommand{\Zd}{\mathbb{Z}_d}
\newcommand{\om}{\omega}
\newcommand{\lam}{\lambda}
\newcommand{\QED}{\hfill $\Box$ \hfill \\}

\newcommand{\divides}{\ \big|\ }

\theoremstyle{plain}
\newtheorem{lem}{Lemma}
\newtheorem{thm}{Theorem}
\newtheorem{theorem}{Theorem}
\newtheorem{cor}{Corollary}
\newtheorem{prop}{Proposition}
\newtheorem{defn}{Definition}
\newtheorem{conj}{Conjecture}
\renewcommand\refname{}
% These colours are tried and tested for titles and headers. 
\usepackage{color}
\definecolor{DarkRed}{rgb}{0.5,0.0,0.0} 
\definecolor{Black}{rgb}{0,0,0} 
\definecolor{DarkBlue}{rgb}{0.1,0.1,0.5} % very dark, almost looks like black
\definecolor{Red}{rgb}{0.9,0.0,0.1}
\definecolor{Blue}{rgb}{0.1,0,0.8} % I designed this for Maple output
\definecolor{Auburn}{rgb}{0.443, 0.184, 0.149}
\definecolor{Saphire}{rgb}{0.0314, 0.145, 0.404}
\definecolor{Burgundy}{rgb}{0.6,0,0.125}
\definecolor{DarkOrchid}{rgb}{0.75,0,0.25}

\definecolor{LightBlue}{rgb}{0.0,0.5,1.0} % I designed this for Background
\definecolor{LightPink}{rgb}{1,0.8,0.8} %background?

\definecolor{Grey}{rgb}{0.5,0.5,0.5} % a very light grey
\definecolor{Green}{rgb}{0,0.5,0.1} % 


% background color
%\pagecolor{LightBlue}

% see documentation for a0poster class for the size options here
% you can define your own headers, vary the colour and size as you like
\let\Textsize\normalsize
\def\Hugehead#1{\noindent{\Huge\color{Saphire} #1}}
\def\LARGEhead#1{\noindent{\LARGE\color{DarkBlue} #1}}
\def\Largehead#1{\noindent{\Large\color{DarkBlue} #1}}
\def\largehead#1{\noindent{\large #1}}
\def\Title#1{\noindent{\VeryHuge\color{DarkBlue} #1}}
   \def\MapleInput#1#2#3{\begin{textblock}{7}(#1,#2) \noindent{\color{Red} $>$ {\bf\tt #3}}\end{textblock}}
\def\MapleOutput#1#2#3{\begin{textblock}{7}(#1,#2){\color{Blue} #3}\end{textblock}}
%\def\MapleOutputC#1{\begin{center}{\color{Blue} #1}\end{center}}

% Set up the grid
%
% Note that [40mm,40mm] is the margin round the edge of the page --
% it is _not_ the grid size. That is always defined as 
% PAGE_WIDTH/HGRID and PAGE_HEIGHT/VGRID. In this case we use
% 15 x 25. This gives us a wide central column for text (7 grid
% spacings) and two narrow columns (3 each) at each side for 
% pictures, separated by 1 grid spacing.
%
% Note however that texblocks can be positioned fractionally as well,
% so really any convenient grid size can be used.

\TPGrid[40mm,40mm]{26}{15}  % 3 - 1 - 7 - 1 - 3 Columns
                            % I am only using two columns: 4.5 - 1 - 9.4
                            % you can do whatever you like

% Mess with these as you like
\parindent=0pt
%\parindent=1cm
\parskip=0.5\baselineskip

\begin{document}

% Understanding textblocks is the key to being able to do a poster in
% LaTeX. In
%
%    \begin{textblock}{wid}(x,y)
%    ...
%    \end{textblock}
%
% the first argument gives the block width in units of the grid
% cells specified above in \TPGrid; the second gives the (x,y)
% position on the grid, with the y axis pointing down.

% You will have to do a lot of previewing to get everything in the 
% right place.

% TOP PART
\begin{textblock}{20}(10,-0.1)
\baselineskip=3\baselineskip\Title{\color{black}\bf Measuring Chaos}
\end{textblock}

\begin{textblock}{6}(20,-0.5)

  \includegraphics[]{mathstatslogohorr}

\Large{\textbf{This research was supported by the Jamie Cassels Undergraduate Research Award}}
\end{textblock}

\begin{textblock}{25.5}(0,0.75)
\begin{center}{\Hugehead{\color{black}\bf Andrew Wilson, Supervised by Dr. Anthony Quas}}
\end{center}
\end{textblock}

%\begin{textblock}{5}(11,1.6)
 % \LARGEhead{{\tt \{anjam\}@uvic.ca}}
%\end{textblock}

% I used xfig to create greyline.eps
\begin{textblock}{26}(0,2.2)
 % \resizebox{26\TPHorizModule}{!}{\includegraphics{greyline.eps}}  
\end{textblock}


\begin{textblock}{7}(0,.3)
 % \resizebox{7\TPHorizModule}{!}{\includegraphics{banner1}} 
  
%\begin{textblock}{7}(0,0)
  %\resizebox{7\TPHorizModule}{!}{\includegraphics{banner2}} 
\end{textblock}

%\begin{textblock}{2}(0,0)
 % \resizebox{1.7\TPHorizModule}{!}{\includegraphics{bigsfulogo.eps}} 
%\end{textblock}

\begin{textblock}{2}(3,1.5)

\end{textblock}

\begin{textblock}{1.8}(21,0.1)
%  \resizebox{3\TPHorizModule}{!}{\includegraphics{CMS.eps}}
\end{textblock}

%%%%% START OF CONTENT %%%%%%%
\begin{textblock}{7}(1,1.8)
\Hugehead{What is Chaos?} %\large
\nocite{*}
\end{textblock}

\begin{textblock}{5.5}(0,2.25)
\Large{Roughly speaking, we say that a system is chaotic if it is sensitive to initial
conditions. That is, in a deterministic
system, if small changes are made to an input, we expect to see vastly different outputs.

A `system' can roughly be thought of as a set of points and a map that moves those points around. 

For example: air molecules in the atmosphere and the wind, or the position of the
end of a double pendulum under the influence of gravity. }
\end{textblock}

\begin{textblock}{2}(5.9,2.8)
\begin{figure}
\vspace{-3 mm}
\hspace{8mm}

% Picture of double pendulum, three stages

\vspace{0 mm}
\includegraphics[scale=1]{./Pictures/doublePendulum.png}
\caption{}
\end{figure}
\end{textblock}

\begin{textblock}{7.5}(0,7)

\Hugehead{Modelling Chaos: Dynamical Systems} 
\smallskip

{\Large In order to study chaos, we need models of the systems we would like to investigate. 
How can we abstract some of the previous examples in a way that can be studied mathematically?}

{\Large {\bf Definition:} A Dynamical System is a pair $(X,T)$ where $X$ is a set of points, and $T$ is 
function from $X$ to $X$.}

rule that describe how a system changes over time

{\Large Formalizing our first example from before, define $$X=\{\text{air particles in the
atmosphere}\}$$ and $T$ to map an air particle to its position after being acted on by the wind for 
one second.} 

{\Large The double pendulum system happens to be a well studied dynamical system. So much so, that 
the map $T$ has been dubbed 'The Standard Map'. (Include Formula?). Below are some illustrations of 
the orbits of points under $T$.}

\end{textblock}

\begin{textblock}{5}(0,13.4)
\large {
\begin{center}
\begin{figure}
%\includegraphics[width=150mm,scale=1]{eq.png}
\vspace{-5mm}
\caption{
Caption
}
\end{figure}
\end{center}
}
\end{textblock}


\begin{textblock}{3.5}(4.5,13.7)
\large{Caption}
\end{textblock}

% end column 1

%%%%%%%%%%%%%%%%%%%%%%%%%%%%%%%%%%%%%%%%%%%%%%%%%%%%%%%%%%%%%%%%
%% SECOND COLUMN of MY POSTER---STARTS here
%%%%%%%%%%%%%%%%%%%%%%%%%%%%%%%%%%%%%%%%%%%%%%%%%%%%%%%%%%%%%%%%

\begin{textblock}{8}(8.4,1.8)
\Hugehead{Stable and Unstable Manifolds}
\end{textblock}

\begin{textblock}{8}(8.4,2.25)
\large{

%The amounts to a facet normal of a cone $C$ in $\R^k$ is a vector $x$ that is perpendicular to a subspace generated by vectors in the generating set of $C$ of dimension $k-1$
}
\end{textblock}

\begin{textblock}{4}(7.8,4)
\large {
\begin{figure}
\vspace{2mm}
\hspace{30mm}
%\begin{tikzpicture}[scale=1.6, transform shape] 
%\node at ( 3.02 , 0.35){\small{$x_1$}};
%\node at (0.55, 3.75){\small{$x_2$}};\
%\draw [-latex, thick] (0,0) -- (2,1);
%\draw [-latex, thick] (0,0) -- (1,3);
%\draw [-latex, thick] (-2,0) -- (3.5,0);
%\draw [-latex, thick] (0,-1.5) -- (0,4);
%\draw [-latex, red, thick] (0,0) -- (-1,2);
%\draw [-latex, red, thick] (0,0) -- (3,-1);
%\fill[gray!80,nearly transparent] (0,0) -- (1.333,4) -- (3,1.5) -- cycle;
%\node at ( 1.25 , 1.5){$\mathcal{C}_A$};
%\coordinate (O) at (0,0);
%\coordinate (A) at (2,1);
%\coordinate (B) at (1,-2);
%\tkzMarkRightAngle[fill=orange,size=0.8cm,opacity=.4](B,O,A);
%\end{tikzpicture}

\vspace{-5mm}
\caption{
caption
}

\end{figure}
}
\end{textblock}

\begin{textblock}{5.3}(11.2,3.9)
\large { textblock}
\end{textblock}

\begin{textblock}{8}(8.4,6.9)
textblock
\end{textblock}

\begin{textblock}{6}(8.4,8.78)
textblock
\end{textblock}

\begin{textblock}{2}(14.6,8.5)
\end{textblock}

\begin{textblock}{8}(8.4,10.7)
\bigskip
\Hugehead{Some Examples of Manifolds}

\large{ Manifolds along the orbit of a periodic point.}

\large{ Manifolds along chaotic point }
\end{textblock}


\begin{textblock}{8}(8.4,13.5)
textblock
\end{textblock}

%% THIRD COLUMN of MY POSTER---STARTS here


\begin{textblock}{8}(16.8,1.8)

\Hugehead{Computing Manifolds(?)/ }

\end{textblock}

\begin{textblock}{5}(16.8,9)
\end{textblock}  

\begin{textblock}{4.2}(22.3,8.8)
\large{textblock}
\end{textblock}

\begin{textblock}{5.5}(16.8,11.1)
\end{textblock} 

\begin{textblock}{3}(23,11.2)
\end{textblock}


\begin{textblock}{9.7}(16.8,13.5)
\LARGEhead{References}
\vspace*{-4cm}
\begin{thebibliography}{99}
%\bibitem{Nash-Will} C.S.J. Nash-Williams. An unsolved problem concerning decomposition of graphs into triangles. \textit{Combinatorial Theory and its Applications III}, pages 1179-1183, 1970.

%\bibitem{Dross} F. Dross, Fractional triangle decompositions in graphs with
large minimum degree, 2015, v3, arXiv:1503.08191.
\end{thebibliography}

\bibliographystyle{unsrt}
\bibliography{facetposter}

\vspace{-5mm}

%%%%% END OF CONTENT %%%%%%%
\end{textblock}

\end{document}
